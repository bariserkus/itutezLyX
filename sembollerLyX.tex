\global\long\def\tabcolsep{1pt}%
\renewcommand\arraystretch{1.1}%

\noindent \hspace{-3mm} %
\begin{longtable}{>{\raggedright}p{1cm}l>{\raggedright}p{13.4cm}}
$a_{0}$ & : & Rayleigh sönüm matrisi için kütle matrisi çarpanı\tabularnewline
$a_{1}$ & : & Rayleigh sönüm matrisi için rijitlik matrisi çarpanı\tabularnewline
$a_{1}$ & : & Menegotto-Pinto modelinde Baushinger malzeme sabitleri\tabularnewline
$a_{2}$ & : & Menegotto-Pinto modelinde Baushinger malzeme sabitleri\tabularnewline
$A_{\text{co}}$ & : & Sheik ve Uzumeri modelinde beton çekirdek alanı\tabularnewline
$A_{\text{s}}$ & : & Sheik-Uzumeri ve Saatçioğlu-Ravzi modellerinde enine donatı alanı\tabularnewline
$A_{\text{sx}}$ & : & Mander modelindende $x$-doğrultusunda uzanan toplam enine donatı
kesit alanı\tabularnewline
$A_{\text{sy}}$ & : & Mander modelindende $y$-doğrultusunda uzanan toplam enine donatı
kesit alanı\tabularnewline
$b$ & : & Mander modelindende enine donatı merkezlerinden ölçülen çekirdek betonunun
$x$-yönüne paralel boyutu\tabularnewline
$b$ & : & Sheik-Uzumeri modelinde çekirdek betonun kenar uzunluğu\tabularnewline
$b_{\text{c}}$ & : & Saatçioğlu-Ravzi modelindende kare kesit çekirdek betonu uzunluğu\tabularnewline
$b_{\text{cx}}$ & : & Saatçioğlu- Ravzi modelindende çekirdek betonun $x$-doğrultusu uzunluğu\tabularnewline
$b_{\text{cy}}$ & : & Saatçioğlu-Ravzi modelindende çekirdek betonun $y$-doğrultusu uzunluğu\tabularnewline
$\mathbf{b}$ & : & Fiber modelde kuvvet interpolasyon fonksiyonu\tabularnewline
$C$ & : & Sheik ve Uzumeri modelinde düşey donatıların eksenleri arasındaki
uzaklık\tabularnewline
$\mathbf{C}$ & : & Sönüm matrisi\tabularnewline
$\mathbf{d}$ & : & Şekil değiştirme matrisi\tabularnewline
$\mathbf{D}$ & : & İç kuvvet matrisi\tabularnewline
$E_{\mathrm{c}}$ & : & Betonun elastisite modülü \tabularnewline
$E_{\text{s}}$ & : & Donatı çeliğinin elastisite modülü\tabularnewline
$E_{\text{sh}}$ & : & Mander çelik modelindende pekleşme modülü\tabularnewline
$F_{\text{y}}$ & : & Akma gerilmesi\tabularnewline
$E_{\text{t}}$ & : & Mander çelik modelindende tanjant elastisite modülü\tabularnewline
$f_{\text{c}}$ & : & Hognestod, Mander ve Kent-Park modellerinde beton basınç gerilmesi\tabularnewline
$f_{\text{c}}^{''}$ & : & Hognestod modelindende betonun maksimum basınç dayanımı\tabularnewline
$f_{\text{c}}^{'}$ & : & Hognestod modelindende beton karakteristik basınç dayanımı\tabularnewline
$\mathit{f}_{\text{c}}'$ & : & Sheik ve Uzumeri modelindende sargılı beton basınç dayanımı\tabularnewline
$f_{\text{cc}}^{'}$ & : & Sheik ve Uzumeri modelindende beton basınç gerilmesi\tabularnewline
$f_{\text{cc}}^{'}$ & : & Mander modelindende sargılı betonda kırılma anında birim şekil değiştirmesi\tabularnewline
$f_{\text{co}}^{'}$ & : & Mander modelindende sargısız betonun basınç dayanımı \tabularnewline
$f_{\text{l}}^{'}$ & : & Mander modelindende ortalama etkili sargılama basıncı\tabularnewline
$f_{\text{lx}}^{'}$ & : & Mander modelindende $x$-doğrultusundaki etkili sargılama basıncı\tabularnewline
$f_{\text{ly}}^{'}$ & : & Mander modelindende $y$-doğrultusundaki etkili sargılama basıncı\tabularnewline
$f_{\text{yh}}$ & : & Mander modelindende enine donatının akma dayanımı\tabularnewline
$f_{\text{yt}}$ & : & Enine donatının akma dayanımı\tabularnewline
$f_{\text{1e}}$ & : & Saatçioğlu ve Ravzi modelindende ortalama yanal sargı basıncı\tabularnewline
$f_{\text{1ex}}$ & : & Saatçioğlu ve Ravzi modelindende $x$-doğrultusunda oluşan etkili
sargı basıncı\tabularnewline
$f_{\text{1ey}}$ & : & Saatçioğlu ve Ravzi modelindende $y$-doğrultusunda oluşan etkili
sargı basıncı\tabularnewline
$f_{\text{yw}}$ & : & Kent-Park modelindende enine donatının akma dayanımı\tabularnewline
$f_{\text{sy}}$ & : & Mander çelik modelindende donatı akma dayanımı\tabularnewline
$f_{\text{su}}$ & : & Mander çelik modelindende donatı nihai dayanımı\tabularnewline
$f_{\text{s}}^{'}$ & : & Sheik ve Uzumeri modelinde enine donatının akma dayanımı\tabularnewline
$\mathbf{f}$ & : & Fleksibilite (esneklik) matrisi\tabularnewline
$\mathbf{F}_{\text{s}}$ & : & Rijitlik tarafından üretilen kuvvet\tabularnewline
$\mathbf{F}_{\text{s},i}^{\text{denge}}$ & : & Dengelenmemiş kuvvet\tabularnewline
$\mathbf{F}_{\text{s},i}^{\text{kabul}}$ & : & Başlangıç rijitliğine bağlı olarak varsayılan kuvvet\tabularnewline
$\mathbf{F}_{\text{s},i}^{\text{içkuv}}$ & : & Bünye fonksiyonlarından elde edilen iç kuvvet\tabularnewline
$h^{'}$ & : & Kent-Park modelinde sargılı beton kısmının etriye dışından etriye
dışına genişliği\tabularnewline
$h$ & : & Mander modelinde enine donatı merkezlerinden ölçülen çekirdek betonunun
$y$-yönüne paralel boyutu\tabularnewline
$\text{k}_{\text{e}}$ & : & Mander modelinde sargılamanın etkinliği ile ilgili katsayı\tabularnewline
$K_{\text{s}}$ & : & Pekleşme rijitliği \tabularnewline
$K_{\text{e}}$ & : & Başlangıç rijitliği\tabularnewline
$\mathbf{K}_{\text{T},i}^{j}$ & : & $i.$ zaman adımı ve $j.$ iterasyon için rijitlik matrisi \tabularnewline
$\mathbf{K}$ & : & Eleman rijitlik matrisi\tabularnewline
$M_{y}$ & : & $y$-eksenine göre eğilme momenti\tabularnewline
$M_{z}$ & : & $z$-eksenine göre eğilme momenti\tabularnewline
$\mathbf{M}$ & : & Kütle matrisi\tabularnewline
$n$ & : & Mander modelinde ve Sheik-Uzumeri modellerinde boyuna donatı sayısı\tabularnewline
$N$ & : & Eksenel kuvveti\tabularnewline
$\mathbf{P}$ & : & Dış kuvvetler\tabularnewline
$\mathbf{Q}$ & : & Serbestlik derecesindeki kuvvetler\tabularnewline
$R$ & : & Menegotto-Pinto modelinde Baushinger etki katsayısı\tabularnewline
$R_{0}$ & : & Menegotto-Pinto modelinde Baushinger malzeme sabitleri\tabularnewline
$s$ & : & Saatçioğlu-Ravzi, Sheik-Uzumeri, Kent-Park modellerinde etriye aralığı\tabularnewline
$s^{'}$ & : & Mander modelinde etriye aralığı\tabularnewline
$u$ & : & $x$-eksenine göre göreceli yer değiştirmelesi\tabularnewline
$\mathbf{u}$ & : & Yer değiştirme matrisi\tabularnewline
$\dot{\mathbf{u}}$ & : & Yer değiştirmenin birinci türevi\tabularnewline
$\ddot{\mathbf{u}}$ & : & Yer değiştirmenin ikinci türevi\tabularnewline
$v$ & : & $z$-eksenine göre göreceli yer değiştirmesi\tabularnewline
$V_{b}$ & : & Yapı taban kesme kuvveti \tabularnewline
$w$ & : & $y$-eksenine göre göreceli yer değiştirmesi\tabularnewline
$w_{\text{i}}$ & : & Mander modelinde düşey donatıların eksenleri arasındaki uzaklık\tabularnewline
$w_{\text{i}}$ & : & Yapının $i$. moduna ait açısal frekansı\tabularnewline
$w_{\text{j}}$ & : & Yapının $j$. moduna ait açısal frekansı\tabularnewline
$\alpha$ & : & Enine donatı ve o doğrultudaki çekirdek betonu arasındaki açı \tabularnewline
$\beta$ & : & Menegotto-Pinto modelinde pekleşme-rijitliği ile elastisite modülü
oranı\tabularnewline
$\beta$ & : & Newmark integrasyon sabiti \tabularnewline
$\rho$ & : & Saatçioğlu ve Ravzi modelinde enine donatının hacimsel oranı\tabularnewline
$\rho_{\text{cc}}$ & : & Mander modelinde toplam boyuna donatının beton çekirdek alanına oranı\tabularnewline
$\rho_{\text{s}}$ & : & Sheik-Uzumeri ve Kent-Park modellerinde enine donatının hacimsel oranı\tabularnewline
$\gamma$ & : & Newmark integrasyon sabiti \tabularnewline
\noindent \textit{$\varepsilon$} & : & Eksenel yer değiştirmeler\tabularnewline
$\varepsilon$ & : & Hognestod modelinde maksimum basınç gerilmesi anındaki birim şekil
değiştirmesi\tabularnewline
$\varepsilon_{\mathrm{c}}$ & : & Hognestod ve Saatçioğlu-Ravzi modellerinde beton birim şekil değiştirmesi\tabularnewline
$\varepsilon_{\text{co}}$ & : & Mander modelinde sargısız betonda maksimum basınç gerilmesi anında
birim şekil değiştirmesi\tabularnewline
$\varepsilon_{\text{cu}}$ & : & Mander modelinde sargılı betonda kırılma anında birim şekil değiştirmesi\tabularnewline
$\varepsilon_{\text{o}}$ & : & Hognestod modelinde maksimum basınç anında birim şekil değiştirmesi\tabularnewline
$\varepsilon_{\text{oo}}$ & : & Sheik ve Uzumeri modelinde maksimum gerilmeye karşılık gelen birim
şekil değiştirmesi \tabularnewline
$\varepsilon_{\text{s}}$ & : & Mander çelik modelinde donatı birim şekil değiştirme değeri\tabularnewline
$\varepsilon_{\text{sh}}$ & : & Mander çelik modelinde donatının pekleşmeye başladığı andaki birim
şekil değiştirmesi\tabularnewline
$\varepsilon_{\text{su}}$ & : & Mander çelik modelinde donatı kopma birim şekil değiştirmesi\tabularnewline
$\varepsilon_{\text{s1}}$ & : & Sheik ve Uzumeri modelinde sargılı betonun maksimum gerilmedeki birim
şekil değiştirmesi\tabularnewline
$\varepsilon_{\text{s2}}$ & : & Sheik ve Uzumeri modelinde sargılı betonun maksimum gerilmede yapabileceği
maksimum birim şekil değiştirmesi\tabularnewline
$\varepsilon_{\text{s85}}$ & : & Sheik ve Uzumeri modelinde sargılı betonun maksimum gerilmesinin \%85'
ine karşılık gelen birim şekil değiştirmesi\tabularnewline
$\varepsilon_{\text{s1}}$ & : & Sheik ve Uzumeri modelinde sargılı betonun maksimum gerilmedeki birim
şekil değiştirmesi\tabularnewline
$\varepsilon_{\text{s2}}$ & : & Sheik ve Uzumeri modelinde sargılı betonun maksimum gerilmede yapabileceği
maksimum birim şekil değiştirmesi\tabularnewline
$\varepsilon_{\text{s85}}$ & : & Sheik ve Uzumeri modelinde sargılı betonun maksimum gerilmesinin \%85'
ine karşılık gelen birim şekil değiştirmesi\tabularnewline
$\varepsilon_{\text{s}}^{*}$ & : & Menegotto-Pinto modelinde donatı birim şekil değiştirmesi\tabularnewline
$\varepsilon_{\text{u}}$ & : & Hognestod modelinde kırılma anında birim şekil değiştirmesi\tabularnewline
$\varepsilon_{\text{y}}$ & : & Mander çelik modelinde donatının akma birim şekil değiştirmesi\tabularnewline
$\varepsilon_{\text{1}}$ & : & Saatçioğlu ve Ravzi modelinde maksimum gerilmeye karşılık gelen birim
şekil değiştirmesi\tabularnewline
$\varepsilon_{\text{01}}$ & : & Saatçioğlu ve Ravzi modelinde sargısız betonda maksimum basınç gerilmesine
karşılık gelen birim şekil değiştirmesi\tabularnewline
$\varepsilon_{\text{50u}}$ & : & Kent-Park modelinde sargısız betona ait gerilmenin sargısız betonun
maksimum gerilmesinin \%50’sine eşit olduğu andaki birim şekil değiştirmesi\tabularnewline
$\varepsilon_{\text{50c}}$ & : & Kent-Park modelinde sargılı betona ait gerilmenin sargılı betonun
maksimum gerilmesinin \%50’sine eşit olduğu durumdaki birim şekil
değiştirmesi\tabularnewline
$\varepsilon_{\text{50h}}$ & : & Kent ve Park modelindende $\mathit{\varepsilon_{\text{50u}}}$ ile
$\varepsilon_{\text{50c}}$ arasındaki birim şekil değiştirme farkı\tabularnewline
$\varepsilon_{\text{085}}$ & : & Saatçioğlu ve Ravzi modelinde sargısız betonda maksimum basınç gerilmesinin
\%85'ine karşılık gelen birim şekil değiştirmesi\tabularnewline
$\varepsilon_{\text{85}}$ & : & Saatçioğlu ve Ravzi modelinde sargılı betonda maksimum basınç gerilmesinin
\%85'ine karşılık gelen birim şekil değiştirmesi\tabularnewline
$\kappa$ & : & Eğrilik\tabularnewline
$\lambda_{\text{c}}$ & : & Mander modelinde sargılı beton dayanımının sargısız beton dayanımına
oranı\tabularnewline
$\xi$ & : & Menegotto-Pinto modelinde son çevrimdeki plastik birim şekil değiştirme
değeri\tabularnewline
$\xi_{i}$ & : & Yapının $i.$ açısal frekansına karşılık gelen sönüm oranı \tabularnewline
$\xi_{j}$ & : & Yapının $j.$ açısal frekansına karşılık gelen sönüm oranı \tabularnewline
$\sigma_{\text{s}}^{*}$ & : & Menegotto-Pinto modelinde donatı akma gerilmesi\tabularnewline
$\sigma_{\text{sa}}$ & : & Menegotto-Pinto modelinde ilk geri yüklemedeki gerilme değeri\tabularnewline
$\sigma_{\text{so}}$ & : & Menegotto-Pinto modelinde ilk yükleme akma noktasındaki gerilme değeri\tabularnewline
\end{longtable}

\noindent 
\global\long\def\tabcolsep{6pt}%
\renewcommand\arraystretch{1.0}%
